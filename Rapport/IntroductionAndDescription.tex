% ICI je vais faire un rappel théorique du projet
% Je vais aussi parler du matériel utilisé, des contrainte et enfin des étapes qu'on a dû réalisé
\section{Introduction}

Notre projet consiste à programmer un robot KUKA de telle sorte qu'il puisse écrire sur un clavier seul.


\subsection{KUKA}

\lettrine[nindent=0em,lines=3]{D}ans le cadre de mon projet d'étude, on a développé un système permettant à un bras mécanique KUKA de taper des choses sur un clavier.
L'automatisation des tâches est devenue essentielle dans de nombreux domaines, et l'utilisation de bras mécaniques offre de vastes possibilités pour améliorer l'efficacité et la précision des processus.
\\
\\
Kuka est avant tout une entreprise de robotique.
C'est une société mondiale d'automatisation.
Le siège de la société se trouve en Allemagne.
%------------------------------------------------

\section{Description des différentes parties du projet}

Lors de ce projet, nous sommes passer par plusieurs étapes.
Voici une description brève des étapes ;

\begin{itemize}
    \item Maintenir une connexion avec le robot.
    \item Configurer ses entrés pour qu'il puisse recevoir des coordonnées.
    \item Programmer en KRL afin qu'il puisse utilisé ses entrés et procéder à des mouvements.
    \item Bouger le robot avec les coordonnées envoyées.
    \item Prendre les coordonnées des touches du clavier.
    \item Bouger le robot pour qu'il puisse taper sur le clavier juste en lui donnant la phrase à taper.
    \item Initialiser les étapes à suivre pour que le robot sache ou appuyer pour réaliser un post.
\end{itemize}


%------------------------------------------------
\section{Matériel et Méthode}

Les outils utilisés sont les suivants ;

\begin{itemize}
    \item Un robot KUKA bien évidement.
    \item Un ordinateur.
    \item Un clavier mécanique.
    \item Un support pour le clavier.
\end{itemize}

En ce qui concerne le language utilisé lors de ce projet.
On a dû utiliser deux languages de programmation, \textbf{Python} et le \textbf{KRL}.
\\
\\
Python a été utilisé pour instancié le serveur.
Ce serveur permet de maintenir la connexion, recevoir/envoyer des coordonnées, vérifier des coordonnées et séquencer la phrase que le robot doit taper.
Les informations que le serveur envoie au robot est sous format \textbf{XML}.
\\
\\
Le KRL permet de se connecter au serveur et d'utiliser les informations que le serveur lui envoie.
Pour que le connexion se fasse, le programme a besoin d'information tels que l'adresse ip et le port du serveur.
Ces informatios se retrouve dans un fichier XML.
Ce fichier possède aussi d'autres informations tels que les données en entrés et en sortie du robot.
\\
\\
Et en ce qui concerne le support de clavier, ce support est obligatoire en vue de respecter la zone de travail du robot.
En effet, le robot possède une zone de travail.
Lorsque le robot est hors de cette zone, le robot est inutillisable.
Cette zone permet d'avoir une sécurité.
Elle permet en autre d'éviter que le robot soit en conatct avec le plan de travail par exemple.

\subsection{Connexion}

YOUSRI

\subsection{Mouvement}

HICHEME

\subsection{Twitter}

HICHEME

