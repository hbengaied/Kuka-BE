%%%%%%%%%%%%%%%%%%%%%%%%%%%%%%%%%%%%%%%%%
% Journal Article
% LaTeX Template
% Version 1.3 (9/9/13)
%
% This template has been downloaded from:
% http://www.LaTeXTemplates.com
%
% Original author:
% Frits Wenneker (http://www.howtotex.com)
%
% License:
% CC BY-NC-SA 3.0 (http://creativecommons.org/licenses/by-nc-sa/3.0/)
%
%%%%%%%%%%%%%%%%%%%%%%%%%%%%%%%%%%%%%%%%%

%----------------------------------------------------------------------------------------
%	PACKAGES AND OTHER DOCUMENT CONFIGURATIONS
%----------------------------------------------------------------------------------------
 
\documentclass[twoside]{article}
\usepackage[francais]{babel}
\usepackage{lipsum} % Package to generate dummy text throughout this template

\usepackage{graphicx}
\usepackage[sc]{mathpazo} % Use the Palatino font
\usepackage[T1]{fontenc} % Use 8-bit encoding that has 256 glyphs
\linespread{1.05} % Line spacing - Palatino needs more space between lines
\usepackage{microtype} % Slightly tweak font spacing for aesthetics

\usepackage[hmarginratio=1:1,top=32mm,columnsep=20pt]{geometry} % Document margins
\usepackage{multicol} % Used for the two-column layout of the document
\usepackage[hang, small,labelfont=bf,up,textfont=it,up]{caption} % Custom captions under/above floats in tables or figures
\usepackage{booktabs} % Horizontal rules in tables
\usepackage{float} % Required for tables and figures in the multi-column environment - they need to be placed in specific locations with the [H] (e.g. \begin{table}[H])
\usepackage{hyperref} % For hyperlinks in the PDF

\usepackage{lettrine} % The lettrine is the first enlarged letter at the beginning of the text
\usepackage{paralist} % Used for the compactitem environment which makes bullet points with less space between them

\usepackage{abstract} % Allows abstract customization
\renewcommand{\abstractnamefont}{\normalfont\bfseries} % Set the "Abstract" text to bold
\renewcommand{\abstracttextfont}{\normalfont\small\itshape} % Set the abstract itself to small italic text

\usepackage{titlesec} % Allows customization of titles
\renewcommand\thesection{\Roman{section}} % Roman numerals for the sections
\renewcommand\thesubsection{\Roman{subsection}} % Roman numerals for subsections
\titleformat{\section}[block]{\large\scshape\centering}{\thesection.}{1em}{} % Change the look of the section titles
\titleformat{\subsection}[block]{\large}{\thesubsection.}{1em}{} % Change the look of the section titles

\usepackage{fancyhdr} % Headers and footers
\pagestyle{fancy} % All pages have headers and footers
\fancyhead{} % Blank out the default header
\fancyfoot{} % Blank out the default footer
\fancyhead[C]{Juin 2023 } % Custom header text
\fancyfoot[RO,LE]{\thepage} % Custom footer text

%----------------------------------------------------------------------------------------
%	TITLE SECTION
%----------------------------------------------------------------------------------------

\title{\vspace{-15mm}\fontsize{24pt}{10pt}\selectfont\textbf{Projet de Bureau d'étude}} % Article title

\author{
\large
\textsc{Yousre Ouali \& Hicheme BEN GAIED}\\
\normalsize HE2B: ISIB \\ % Your institution
\normalsize \href{mailto:54667@etu.he2b.be}{54667@etu.he2b.be} \& \href{mailto:58930@etu.he2b.be}{58930@etu.he2b.be} % Your email address
\vspace{-5mm}
}
\date{}

%----------------------------------------------------------------------------------------

\begin{document}

\maketitle % Insert title

\thispagestyle{fancy} % All pages have headers and footers

%----------------------------------------------------------------------------------------
%	ABSTRACT
%----------------------------------------------------------------------------------------

\begin{abstract}

\noindent L'objectif de notre projet est de programmer le robot KUKA afin qu'il puisse taper sur un clavier. On a décidé de faire un compte Twitter au robot et lui faire Twitter des phrases. Ces phrases auront pour sujet Pokemon. Il nous dira quelles sont les évolutions des Pokemon.

\end{abstract}

%----------------------------------------------------------------------------------------
%	ARTICLE CONTENTS
%----------------------------------------------------------------------------------------
\textbf{Mots-clefs :} Robot, KUKA, Clavier, Twitter, Pokemon, Souris, Bras mécanique, KRL, Python
\begin{multicols}{2} % Two-column layout throughout the main article text

% ICI je vais faire un rappel théorique du projet
% Je vais aussi parler du matériel utilisé, des contrainte et enfin des étapes qu'on a dû réalisé
\section{Introduction}

Notre projet consiste à programmer un robot KUKA de telle sorte qu'il puisse écrire sur un clavier seul.


\subsection{KUKA}

\lettrine[nindent=0em,lines=3]{D}ans le cadre de mon projet d'étude, on a développé un système permettant à un bras mécanique KUKA de taper des choses sur un clavier.
L'automatisation des tâches est devenue essentielle dans de nombreux domaines, et l'utilisation de bras mécaniques offre de vastes possibilités pour améliorer l'efficacité et la précision des processus.
\\
\\
Kuka est avant tout une entreprise de robotique.
C'est une société mondiale d'automatisation.
Le siège de la société se trouve en Allemagne.
%------------------------------------------------

\section{Description des différentes parties du projet}

Lors de ce projet, nous sommes passer par plusieurs étapes.
Voici une description brève des étapes ;

\begin{itemize}
    \item Maintenir une connexion avec le robot.
    \item Configurer ses entrés pour qu'il puisse recevoir des coordonnées.
    \item Programmer en KRL afin qu'il puisse utilisé ses entrés et procéder à des mouvements.
    \item Bouger le robot avec les coordonnées envoyées.
    \item Prendre les coordonnées des touches du clavier.
    \item Bouger le robot pour qu'il puisse taper sur le clavier juste en lui donnant la phrase à taper.
    \item Initialiser les étapes à suivre pour que le robot sache ou appuyer pour réaliser un post.
\end{itemize}


%------------------------------------------------
\section{Matériel et Méthode}

Les outils utilisés sont les suivants ;

\begin{itemize}
    \item Un robot KUKA
    \item Un ordinateur
    \item Un clavier mécanique
    \item Un support pour le clavier
\end{itemize}

En ce qui concerne le language utilisé lors de ce projet.
On a dû utiliser deux languages de programmation, \textbf{Python} et le \textbf{KRL}.
\\
\\
Python a été utilisé pour instancié le serveur.
Ce serveur permet de maintenir la connexion, recevoir/envoyer des coordonnées, vérifier des coordonnées et séquencer la phrase que le robot doit taper.
Les informations que le serveur envoie au robot est sous format \textbf{XML}.
\\
\\
Le KRL permet de se connecter au serveur et d'utiliser les informations que le serveur lui envoie.
Pour que le connexion se fasse, le programme a besoin d'information tels que l'adresse ip et le port du serveur.
Ces informatios se retrouve dans un fichier XML.
Ce fichier possède aussi d'autres informations tels que les données en entrés et en sortie du robot.
\\
\\
Et en ce qui concerne le support de clavier, ce support est obligatoire en vue de respecter la zone de travail du robot.
En effet, le robot possède une zone de travail.
Lorsque le robot est hors de cette zone, le robot est inutillisable.
Cette zone permet d'avoir une sécurité.
Elle permet en autre d'éviter que le robot soit en conatct avec le plan de travail par exemple.

\subsection{Connexion}

En ce qui concerne la connexion.
Le robot est connecté en \textbf{VLAN}.
Donc, il n'est pas directement lié au réseau.
Pour pouvoir communiquer avec celui-ci, on est obligé de se connecte au VLAN.
\\
\\
Lorsque la communication entre le serveur et le robot est assurés, on peut passer à l'étape suivante qui consiste à créer un fichier XML.
Ce fichier XML contiendra toute les informations afin d'assurer la connexion.
Comme par exemple, l'adresse ip du serveur, le port que le serveur utilise et le protocol utilisé pour communiquer.
\\
\\
Et enfin, lorsque le fichier XML est bien configuré, on peut passer au programme KRL.
On a crée un objet \textbf{RSI} de type \textbf{ST\_ETHERNET}.
Cette objet spécifique au language \textbf{KRL} permettra d'initialisé la connexion.
Lorsque la connexion a été établie, le robot commence par envoyé le premier fichier XML au serveur.
Ce premier fichier contiendra des informations comme les coordonnées, mais aussi un \textbf{IPOC}.
Cette valeur est composée de chiffre.
Elle devra être renvoyé au robot selon un temps impartie.
Si cette valeur n'est pas renvoyé dans les plus bréves délais, alors la connexion s'interromt.

\subsection{Configuration du fichier XML}

Comme expliqué précédement, le fichier XML permet de configuré la connexion du robot au serveur.
Toutefois, ce fichier contient d'autre informations.
Voyons voir de plus près l'architecture de ce fichier.
\\
\\
La balise racine de ce fichier se nomme "\textbf{ROOT}".
Dans la balise racine se trouve 3 autres balises.
Il y a une balise qui se nomme "\textbf{CONFIG}", une autre "\textbf{SEND}" et la dernière s'appelle "\textbf{RECEIVE}".
\\
\\
La balise \textbf{CONFIG} va permettre de configurer les paramètres de communication.
On va y retrouver 6 autres balises ;
\begin{itemize}
    \item "\textbf{IP\_NUMBER}" contenant l'addresse IP du serveur.
    \item "\textbf{PORT}" permettant d'avoir le port utilisé par le serveur.
    \item "\textbf{PROTOCOL}" qui défini le protocole utilisé pour la communication. On a deux types de protocoles utilisable, on a le droit d'utiliser le protocole \textbf{TCP} ou \textbf{UDP}.
    \item "\textbf{SENDTYPE}" est l'identifieur du système externe. Cette valeur va être identifier pour chaques paquets reçu au robot.
    \item "\textbf{PROTOCOLLENGTH}" est une balise permettant d'activier la transmission de la longueur de byte du protocole. Elle n'a que deux valeurs possibles ("\textbf{ON}" ou "\textbf{OFF}").
    \item Et enfin, la balise "\textbf{ONLYSEND}" permet de définir la direction des échanges de donnée. Elle peut être mis à "\textbf{TRUE}", ceci permet au robot d'envoyer des données mais pas d'en recevoir. Tandis que, lorsqu'il est mis à "\textbf{FALSE}", alors le système peut envoyer et recevoir des données. 
\end{itemize}
Concernant la balise \textbf{PROTOCOLLENGTH}, il a été initialisé à \textbf{OFF} et la balise \textbf{ONLYSEND} a été initialisé à \textbf{FALSE}.
\\
\\
Ensuite, la balise \textbf{SEND} permet de définir ce que le robot peut envoyer.
Cette balise contient une sous-balise "\textbf{ELEMENTS}" qui en contient aussi une autres sous-balise "\textbf{ELEMENT}".
La balise \textbf{ELEMENT} a des attributs tels que ;
\begin{itemize}
    \item \textbf{TAG} est le nom du tag qui sera généré par le robot.
    \item \textbf{TYPE} est le type de donnée.
    \item \textbf{INDX} est le numéro de sorite de l'objet.
    \item Et enfin, \textbf{UNIT} est l'unité utilisé pour la donnée.
\end{itemize}
Pour ce projet, j'ai juste besoin des coordonnées du robot.
Le système possède déjà un objet préconfiguré qui se nomme "\textbf{DEF\_RIst}".
C'est dans l'attribut \textbf{TAG} où on précise le nom de cette objet.
Le type de cette objet est un \textbf{DOUBLE}, le numéro de sortie est \textbf{INTERNAL} et l'unité est \textbf{0}.
\\
\\
Remarque, il existe plein d'autres mots-clés comme \textbf{DEF\_RSol} (envoie les commandes de positions cartésienne), ou le \textbf{DEF\_MACur} (envoie les courants moteur du robot de l'axe A1 à A6).
\\
\\
Et enfin, la balise \textbf{RECEIVE} est la balise qui définie la sortie du robot.
Cette balise à la même architecture que la balise \textbf{SEND}.
Toutefois, la balise \textbf{ELEMENT} contient un attribut en plus, c'est le \textbf{HOLDON}.
Ce nouveau attribut va définir le comportement de l'objet reçu au robot lorsque celle-ci à une erreur.
Elle a deux valeurs possibles ;
\begin{itemize}
    \item \textbf{0} permet de réinitialisé la veleurs reçu.
    \item \textbf{1} permet de maintenir l'ancienne valeur valide alors que la nouvelle valeur a une erreur.
\end{itemize}
Dans ce projet, tous les eléments ont un \textbf{HOLDON} à \textbf{1} afin d'assurer une localisation toujour valide.
\\
\\
Passons maintenant dans le language KRL,

\subsection{Mouvement}

HICHEME

\subsection{Twitter}

HICHEME



%------------------------------------------------

\section{Problème rencontré}

Durant notre projet, nous avons rencontré un problème lié à l'utilisation de la souris par le bras.
En effet, notre objectif était d'utiliser la souris pour déplacer le curseur vers le bouton "Tweet" afin de soumettre le tweet rédigé par le robot, au lieu d'utiliser les huit tabulations.
Pour cela, nous avons développé un boîtier adapté à la tête du robot et implémenté le code nécessaire pour cette fonctionnalité.
Lors de nos tests, cela fonctionnait très bien, mais lorsque nous sommes passés en mode T2, qui permet au robot d'aller plus vite, un problème majeur et évident est apparu.
La distance parcourue par le curseur variait en fonction de la vitesse.
Malheureusement, nous n'avions pas eu la possibilité de calibrer la souris avec le mode T2, car nous ne disposions pas de la clé nécessaire pour activer ce mode.
Nous pensons toutefois qu'il est possible de résoudre ce problème en ajoutant du code dans le fichier contenant les instructions KRL, de manière à imposer une vitesse au robot lors de l'utilisation de la souris.
\\
\\
Cependant, par manque de temps, nous avons finalement opté pour la méthode des
tabulations et de la touche Enter afin de publier nos tweets.

\section{Résultat final}

Le résultat final de notre projet est que le robot KUKA est capable de taper sur les touches du clavier en fonction de la phrase fournie par le système externe.
Dans notre cas, nous avons mis en place une configuration où le serveur (système externe) effectue une requête à une API spécifique.
Cette API fournit des informations sur les évolutions des Pokémon.
Ainsi, lorsque le serveur reçoit une requête pour une évolution spécifique, il transmet cette information au robot KUKA qui exécute les mouvements nécessaires pour taper la phrase correspondante sur le clavier.
Cela permet d'automatiser le processus d'envoi de commandes liées aux évolutions des Pokémon en utilisant le robot.
\\
\\
En plus de pouvoir taper sur les touches du clavier, le robot a également la capacité de déplacer la souris pour soumettre un Tweet.
Cependant, ce type de mouvement est limité au niveau T1 du bras mécanique.

\section{Améliorations possibles}

Il y a plusieurs améliorations possibles pour le projet, notamment :
\begin{enumerate}
    \item Équiper le robot d'une caméra pour reconnaître les lettres du clavier : Cela permettrait d'éviter d'avoir à spécifier les coordonnées de chaque lettre individuellement. Le robot serait capable de reconnaître les lettres visuellement et d'interagir avec le clavier de manière plus intuitive.
    \item Ajouter un moteur au bout du robot pour une frappe plus rapide : En équipant le bout du robot d'un moteur, il pourrait effectuer des mouvements de frappe sur les touches du clavier de manière plus rapide et précise. Cela éliminerait la nécessité de mouvements complexes sur l'axe Z, ce qui accélérerait le processus de rédaction de la phrase.
    \item Générer des phrases avec une IA : Cette amélioration a été envisagée, mais il y a eu des obstacles liés aux coûts. L'idée était d'utiliser une IA pour générer des phrases de manière automatique, offrant ainsi une variété de sujets et de phrases uniques à chaque tweet. Cependant, l'utilisation d'une IA de ce type nécessite souvent un quota de requêtes payantes.
    \item Équilibrer le mouvement de la souris avec le niveau T2 du robot KUKA : L'amélioration serait de calibrer et de synchroniser les mouvements de la souris avec le mode T2 du robot. Cela permettrait au robot de se déplacer plus rapidement tout en maintenant une précision suffisante lors de l'utilisation de la souris pour des actions telles que cliquer sur des boutons ou naviguer dans une interface graphique.
    \item Permettre au robot de reconnaître sa position sur l'écran, ce qui lui permettrait de déplacer la souris de manière autonome. Cela impliquerait d'intégrer des fonctionnalités de vision par ordinateur et de traitement d'image au système du robot. En utilisant des techniques telles que la détection d'objets et la reconnaissance de motifs, le robot serait capable de localiser sa position relative sur l'écran et de naviguer avec précision. Cela faciliterait l'interaction du robot avec des interfaces graphiques complexes, lui permettant d'effectuer des actions spécifiques à des emplacements précis sur l'écran sans nécessiter de coordonnées préprogrammées.
\end{enumerate}

Ces améliorations permettraient d'optimiser davantage le projet en termes de vitesse, précision et automatisation des tâches liées à l'utilisation du clavier et de la souris par le robot KUKA.
\begin{thebibliography}{99}


    % EXEMPLE

    \bibitem[Pozyx]{}
    2022
    
    \newblock UWB is here to stay
    
    \url{https://www.pozyx.io/technology/uwb-technology#UWB-is-here-to-stay} 
    
    
    % \bibitem[Université de Mons]{}
    % 2015-2016
    
    % \newblock Maximilien CHARLIER
    
    % \newblock  Technologie Ultra Wide Band dans l’Internet des Objets
    
    % \bibitem[Stack Overflow contributors]{}
    
    % \newblock Learning MQTT
    % \url{https://riptutorial.com/Download/mqtt.pdf}
    
    
    % \bibitem[MDN Web Docs]{}
    % 16 mars 2022
    
    % \newblock Manipuler des données JSON
    
    % \url{https://developer.mozilla.org/fr/docs/Learn/JavaScript/Objects/JSON}
    
    \end{thebibliography}
    

\end{multicols}

\end{document}
